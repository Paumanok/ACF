\documentclass[12pt]{article}

\usepackage{graphicx} % pictures
\usepackage{multicol} %Make columns
\usepackage{multirow} %Make rows
\usepackage{amsmath} % Mathy things
\usepackage{gensymb} % Literally only need for degree symbol why 
\usepackage{amssymb}
\usepackage{sympytex} 
\usepackage{booktabs} % Pretty Tables
\usepackage{colortbl} % Might replace xcolor for table  coloring
\usepackage{array} % Tably stuff
\usepackage{longtable} 
\usepackage[margin=1in]{geometry} %Change margins
\usepackage{fancyhdr} 
\usepackage{tabu} 
\usepackage{tabularx} 
\usepackage{pdfpages} % include pdf
\usepackage{pgfplots}
\usepackage{pgfplotstable}
\usepackage[per-mode=symbol]{siunitx}
\usepackage{float}
\usepackage[labelfont=bf,textfont=small]{caption} %Makes captions look nice, will use in the future (maybe)

\usepackage{tocloft}

\renewcommand{\figurename}{Fig.}
\def\cequationautorefname{Equation}%
\usepackage{hyperref}
\usepackage{listings}
\pagestyle{fancy}
\fancyhead{}
\fancyfoot{}
\rhead{\hfill\thepage}
\lhead{\projectDescriptionShort}
\lstset{%
    basicstyle={\scriptsize\ttfamily},
    inputpath=./lvsreps
    }

\definecolor{lightgray2}{gray}{0.9} % Color for table rows

\newcolumntype{x}[1]{>{\centering\arraybackslash}p{#1}} %center fixed with columns 
%Use x{''width''}
\newfloat{cequation}{H}{equ}
\floatname{cequation}{Equation}
\newcounter{NameOfTheNewCounter}
\newcommand{\startalign}{\setcounter{NameOfTheNewCounter}{\theequation+1}}
\newcommand{\equationset}[1]{% \equationset{<caption>}
\noindent\makebox[\linewidth]{Equations \theNameOfTheNewCounter-\theequation: \textbf{#1}}\bigskip}% Print caption
%allow align environment numbering
\newcounter{engineering}
\renewcommand{\theengineering}{\Alph{engineering}}
\newcommand\engineerthis[1]{\refstepcounter{engineering}\label{#1}\theengineering. }
\newcommand\numberthis{\addtocounter{equation}{1}\tag{\theequation}}
\newcommand\Vcc{\ensuremath{V_{cc}}}
\newcommand\sip{\texttt{SI}}
\newcommand\clk{\texttt{clk}}
\newcommand\ao{\texttt{AO}}
\newcommand\mm{~\si{\milli\meter}}
\newcommand{\kohm}[1]{#1~\si{\kilo\ohm}}
\newcommand{\khz}[1]{#1~\si{\kilo\hertz}}

\setlength{\parindent}{0pt}
\setlength{\parskip}{6pt}

\graphicspath{{images/}}
\pgfplotsset{compat=1.13}
\pgfplotstableset{%
    %every head row/.style={before row=\toprule,after row=\midrule},
    every last row/.style ={after row=\bottomrule},
    every even row/.style={before row={\rowcolor[gray]{0.9}}},
}
% The following should be changed to represent your personal information
\newcommand{\projectDescription}{A smart solution to automate and regulate \\the feeding of common household pets}
\newcommand{\projectTitle}{Automated Pet Feeder}
\newcommand{\yourname}{Chris Ranc}
\newcommand{\myname}{Matt Smith}
\newcommand{\anothername}{Shabab Siddiq}
\newcommand{\dateSubmitted}{November 7, 2017}
\newcommand{\collab}{In Collaboration with teams from Northeastern University, Georgia Tech, and Virginia Tech}
\newcommand{\componentDescription}{Microcontroller Options}
\newcommand{\projectDescriptionShort}{Microcontroller Options}


\setcounter{secnumdepth}{0} % Remove section numbering

\begin{document}
    
\thispagestyle{empty}
    \vspace*{2.5cm} 
    \begin{center}
        \LARGE
        \textbf{\projectTitle}

        \Large
        \projectDescription

    \vspace*{2.5cm} 
        \large
        \textbf{CMPE 495 Independent Risk Investigation}

        \componentDescription
    \end{center}
    
    \vspace*{2cm}
    
    \begin{multicols}{2}
        \phantom{LaTeX doesn't like empty columns} % Phantom will take up that much space, but not actually appear
        \columnbreak{}
        \begin{raggedright}
            
        Name: \yourname\\
        Team Members: \myname\
        \phantom{Team Members:} \anothername\\
        Submitted: \dateSubmitted\\
        \vspace{\baselineskip}
        \end{raggedright}
    \end{multicols}
\newpage
%
% TOC
%
\renewcommand{\cftaftertoctitle}{\thispagestyle{empty}} 
\renewcommand\cftsecleader{\cftdotfill{\cftdotsep}}
\tableofcontents
\newpage

\section{Overview}
%Brief textual description of the role of this high-risk component in the overall project
Like any embedded system a microcontroller that is cost effective and provides adequate hardware
for the requirements of the system is needed.  These requirements being to operate the mechanical 
communication aspects of the feeder.  It must have available documentation and easy to use
toolchains, with useful SDKs, for efficient development.  
%An engine controller, like any electrical device requires power to function properly. 
% The system is heavily constrained by both weight and size requirements and thus must be 
% developed carefully to ensure compliance. The power systems will be responsible for delivering 
% power to both the electrical components, such as microcontrollers, and mechanical components, 
% like valves. 
\section{Risk specification}
Marketing requirements for microcontrollers:
\begin{enumerate}
    \item\label{light} Must be lightweight.
    \item\label{remote}Must be able to control remote vent valves for propellant and pressurant tanks.
    \item\label{depress}Must be able to depressurize the rocket independent of the launch controller.
    \item\label{pressure}Must have remote electronic pressure instrumentation for tank pressures.
    \item\label{temp}Must have remote electronic temperature instrumentation for tank temperatures.
    \item\label{ignition}Must ignite only from an electrical ignition source with a key lock-out on the pad and with the same key lock-out at the main launch controller.
    \item\label{press}Must be able to open the helium press valve before and keep it open during flight.
    \item\label{pyro}Must be able to fire the main pyro valves for takeoff.
    \item\label{spark}Must be safe to operate in a flammable environment.
\end{enumerate}
The relevant specifications can be found in \autoref{tab:specs}.

\begin{table}[H]
    \centering
    \caption{Engineering Specifications}
    \label{tab:specs}
    \begin{tabularx}{\linewidth}{cXX} \toprule
        Marketing Requirement(s) & Engineering Requirement & Justification \\ \midrule
        \ref{light}              & \engineerthis{elight}Must weigh less than 1 lb& The rocket cannot weigh too much or the height goal is not achievable.\\
        \ref{remote},\ref{depress}              & \engineerthis{eremote}Must be able to open the release valves & The relief valves allow the launch to be scrubbed, as well as the air released during filling.\\
        \ref{pressure}              & \engineerthis{epressure}Must be able to read from a pressure transducer & The propulsion team needs to be able to monitor tank pressure before liftoff to determine whether to scrub the launch.\\
        \ref{temp}              & \engineerthis{etemp}Must be able to read from a thermocouple& The propulsion team needs to monitor tank temperature to determine whether to scrub the launch.\\
        \ref{ignition}              & \engineerthis{eigintion}Must not ignite the engine& The rocket launch, per range safety rules, must be initialized by a keyed system.\\
        \ref{press}              & \engineerthis{epress}Must be able to power the press valve for at least 30 minutes at 24~\si{\volt} and 0.4~\si{\ampere}& The helium valve is opened on the ground and may need to remain so for an extended period of time.\\
        \ref{pyro}              & \engineerthis{epyro}Must be able to power the two pyro valves for 1 second at 24~\si{\volt} and 1~\si{\ampere} & The pyro valves are the main engine valves the release the propellants into the chamber and cannot be closed via electrical means.\\
        \ref{spark}              & \engineerthis{espark}Must not produce any sparks during operation & Sparking around flammable materials in a high oxygen environment could lead to RUD (rapid unscheduled disassembly)\\
    \end{tabularx}
\end{table}
Many of these specifications, such as \ref{elight}, do not appear directly connected to the power system, but must be kept in mind, as they make options such as more batteries infeasible.
%Mostly tables introduced and explained in text
    %– Needs (0–2 points)—numbered customer needs or marketing requirements related to
    %this risk, (i.e., a subset of the overall project needs)
    %– Engineering specifications (0–2 points)—numbered/lettered engineering specification
    %related to this risk, each related to needs by number, (i.e., a subset of the overall project
    %specifications).
    %– Analysis to justify engineering specifications (0–3 points)
\section{Risk investigation}
%Combination of textual explanation and tables
    %– Survey of existing systems/components and how they relate to this risk (0–5 points)
    %– Concepts considered and chosen (0–3 points)
    %– Rationale for choice(s), including analysis performed, (e.g., Pugh table) (0–4 points)
\section{Risk mitigation design}
%Technical details of proposed design choice—textual explanation with many meaningful
%diagrams, (introduced and explained in text); should include input(s) and output(s) and
%should define any interfaces/protocols
    %– Overview (0–2 points)
    %– How it works (0–10 points)—include specifications for this component to enable the
    %overall project to meet its engineering specifications, (which were presented in Section
    %II)
    %– Why this design mitigates, (i.e., eliminates or minimizes), risk (0–10 points)
    %– Overall risk subsystem diagrams (0–5 points)
    %– Detailed design (0–5 points)
        %· Circuits and assemblies, of any hardware components
            %o Circuit schematics
            %o Sensor types and placement
            %o Actuators
            %o Other components
            %o Interface(s) to other project components
            %o State diagrams, flowcharts, system models, etc.
        %· User interface and controls, of any software components
            %o State diagrams, flowcharts, system models, etc.
    %– Intellectual property (0–3 points)—results of patent search for design of proposed risk
    %solution
\section{Parts List}
%List of parts for proposed risk mitigation design.
    %– Part description (with name and number, if applicable)
    %– Cost—(standard purchase price of component)
    %– Your cost—(your cost for component; could be $0 if you already own)
    %– Availability—(vendor information, including lead time from order placement to delivery)
\section{Testing strategy}
%Plan for verification of proper component/subsystem operation within the project’s
%particular operating environment. The plan should include what will be measured/tested
%along with an estimate of when. (Notes: 1. manufacturer’s performance specifications are
%often unachievable in practice, [e.g., new car mileage estimates]; 2. testing sooner is much
%better than later— whether this high-risk component will perform adequately in the overall
%project system needs to be determined as early as possible.)
\section{Uncertainties}
%Any remaining doubts or questions about using or testing this component/subsystem—any
%concerns the author or other team members have about this aspect of the project.
\section{Appendices}
%Copies of (or links to) the manufacturer’s literature on the component/subsystem—
%especially technical specification(s) and application notes.

\end{document}
